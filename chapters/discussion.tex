%
\chapter{Discussion}
\label{ch:discussion}






% Flatness of the specimen surface
% check out: Newbury and Ritchie 2013, Quantitative SEM/EDS, Step 1: What Constitutes a Sufficiently Flat Specimen?
% https://www.cambridge.org/core/services/aop-cambridge-core/content/view/E9E18A67EED08A3A7F23C4559F81DE93/S1431927613008210a.pdf/quantitative-semeds-step-1-what-constitutes-a-sufficiently-flat-specimen.pdf


% Sb M-lines not in any table, but M is stated to be 0.733 keV at https://www.globalsino.com/EM/page4675.html




\section{Regarding the ISO 15632 standard}
\label{theory:eds_performance:iso15632}

BAM (Federal Institute for Materials Research and Testing) has a test sample for SEM EDS with a accompanying software, which satisfies the ISO 15632 standard.
However, the price per 28.02.2023 for this sample (EDS TM002) is 334 EUR and 335 EUR for the software.
A full test in accordance with the ISO standard could probably be done with a cheaper sample and a Jupyter Notebook, where the user would see all the steps and the results.
This has not been a main focus in this work, but some parameters in the ISO standard are covered in this work.

\url{https://webshop.bam.de/webshop_de/eds-tm002.html}

paper at \url{https://www.cambridge.org/core/services/aop-cambridge-core/content/view/17A1D769BB6B9B76B5AC815911FAE7FD/S1431927613008271a.pdf/check-and-specification-of-the-performance-of-eds-systems-attached-to-the-sem-by-means-of-a-new-test-material-eds-tm002-and-an-updated-evaluation-software-package-eds-spectrometer-test-version-3-4.pdf}






% TODO: energy resolution. the older xmax info claims high energy resolution and high cps. The x-max-N datasheet lists resolution with cps, which is nicer. Progress.








% \section{Qualitative analysis}
% \label{method:qualitative_analysis}

% See Annex A in ISO 22309. Also make a notebook, which can be shared with HyperSpy people.
% Show the use of find_peak_ohaver, refering to O'Haver's paper, with different settings.
% Also show nice plotly plots, eventually plt with plotting all lines in HS for the element.
% And maybe also fitting, where the area is shown to see if any added elements are 0.
% Can refer to autoID, and it's flaws.

% Rule for what a peak is? See eg Annex A in ISO 22309: I > (BG + 3 x sqrt(BG)), wher BG is mean of BG (probably at the given energy)

% \brynjar{ISO 22309: "typically a value of 30 eV can be critical; peaks separated by more than this figure should not be confused by either automatic or manual identification procedures."}






% \subsection{Calibration}
% \label{method:calibration}

% For some reason, it works better to calibrate the offset and then the scale/dispersion.
% HyperSpy allows for calibration of offset, scale or energy resolution, but only one at a time.
% The procedure used was: calibrate offset, calibrate scale, recaliabrate offset, and recalibrate scale.
% After the recalibration, the difference was compared to the first calibration, and if the change was more than 5\%, the calibration was repeated.
% By default, the HyperSpy calibration uses all alpha lines.
% It is possible to specify certain lines to calibrate on, but this yeilded the same or worse results as using all alpha lines.
% \brynjar{The stuff above is more discussion, at least the last sentence.}
