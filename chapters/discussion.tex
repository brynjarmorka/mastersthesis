%
\chapter{Discussion}
\label{ch:discussion}






% Flatness of the specimen surface
% check out: Newbury and Ritchie 2013, Quantitative SEM/EDS, Step 1: What Constitutes a Sufficiently Flat Specimen?
% https://www.cambridge.org/core/services/aop-cambridge-core/content/view/E9E18A67EED08A3A7F23C4559F81DE93/S1431927613008210a.pdf/quantitative-semeds-step-1-what-constitutes-a-sufficiently-flat-specimen.pdf


% Sb M-lines not in any table, but M is stated to be 0.733 keV at https://www.globalsino.com/EM/page4675.html




\section{Regarding the ISO 15632 standard}
\label{theory:eds_performance:iso15632}

BAM (Federal Institute for Materials Research and Testing) has a test sample for SEM EDS with a accompanying software, which satisfies the ISO 15632 standard.
However, the price per 28.02.2023 for this sample (EDS TM002) is 334 EUR and 335 EUR for the software.
A full test in accordance with the ISO standard could probably be done with a cheaper sample and a Jupyter Notebook, where the user would see all the steps and the results.
This has not been a main focus in this work, but some parameters in the ISO standard are covered in this work.

\url{https://webshop.bam.de/webshop_de/eds-tm002.html}

paper at \url{https://www.cambridge.org/core/services/aop-cambridge-core/content/view/17A1D769BB6B9B76B5AC815911FAE7FD/S1431927613008271a.pdf/check-and-specification-of-the-performance-of-eds-systems-attached-to-the-sem-by-means-of-a-new-test-material-eds-tm002-and-an-updated-evaluation-software-package-eds-spectrometer-test-version-3-4.pdf}






% TODO: energy resolution. the older xmax info claims high energy resolution and high cps. The x-max-N datasheet lists resolution with cps, which is nicer. Progress.








% \section{Qualitative analysis}
% \label{method:qualitative_analysis}

% See Annex A in ISO 22309. Also make a notebook, which can be shared with HyperSpy people.
% Show the use of find_peak_ohaver, refering to O'Haver's paper, with different settings.
% Also show nice plotly plots, eventually plt with plotting all lines in HS for the element.
% And maybe also fitting, where the area is shown to see if any added elements are 0.
% Can refer to autoID, and it's flaws.

% Rule for what a peak is? See eg Annex A in ISO 22309: I > (BG + 3 x sqrt(BG)), wher BG is mean of BG (probably at the given energy)

% \brynjar{ISO 22309: "typically a value of 30 eV can be critical; peaks separated by more than this figure should not be confused by either automatic or manual identification procedures."}






% \subsection{Calibration}
% \label{method:calibration}

% For some reason, it works better to calibrate the offset and then the scale/dispersion.
% HyperSpy allows for calibration of offset, scale or energy resolution, but only one at a time.
% The procedure used was: calibrate offset, calibrate scale, recaliabrate offset, and recalibrate scale.
% After the recalibration, the difference was compared to the first calibration, and if the change was more than 5\%, the calibration was repeated.
% By default, the HyperSpy calibration uses all alpha lines.
% It is possible to specify certain lines to calibrate on, but this yeilded the same or worse results as using all alpha lines.
% \brynjar{The stuff above is more discussion, at least the last sentence.}






% \subsection{The requirements for a standard material}
% \label{theory:eds_performance:standard_material}
% smt about the standard materials available \dots
% what we want in the spectrum \dots
% candidates \dots







% \subsection{Other tests - not used in this work}
% \label{theory:eds_performance:other}

% \brynjar{TODO from Ton: "if not used, do not include them. But consider to use them in discussion leading to concrete future work (last chapter)"}

% Other test are described in the literature, but they were not used in this work.
% Some of them is mentioned here, either because they are interesting but not possible, because they are not relevant enough, or because they could be relevant in future works.

% \subsubsection*{Linearity and stability}
% In Goldstein \cite[p. 232]{goldstein_scanning_2018} it is stated that the two most important tests for an EDS detector are linearity and stability.
% This is stated in the section about what to look for when buying a detector.
% The reason for not including these two test in the work is mainly their dependence on a way to measure the probe current, which e.g. can be done with a Faraday cup.
% Such tests could be used in the future and are therefore briefly described here.
% Linearity of the detector is that the number of X-rays measured is proportional to the number of X-rays generated.
% Stability is that the detector resolution and the peak positions does not change significantly with different probe currents.
% Both tests require multiple spectra of the same sample with different probe currents.
% The ISO 22309 standard on quantification of EDS spectra \cite{iso_quantification_22309} also mentions these two tests.

% % \brynjar{ISO 22309: "Whatever type of detector is being used, the count rate capabilities of the system should be checked by
% %     comparing a spectrum obtained at a count rate below 2 000 counts/s with a spectrum obtained at the highest
% %     count rate to be used, to look for peak shift and pile-up distortion that might affect relative peak heights. A
% %     minimum of two checks on beam stability, using a Faraday cup, or a known reference specimen, should be
% %     made prior to and following the analysis."}

% % TODO : kan jeg gjøre de med beam current???



% \subsubsection*{Peak shape}

% Deviations of peak shapes can be used to identify problems with the detector or the qualitative analysis.
% Peak shapes are used in Egerton and Chengs work to identify incomplete charge collection, but this is a lesser concern in modern SDDs.
% Peak shapes deviating from a Gaussian shape, which is not due to detector issues, are probably due to misidentification where a minor peak is not correctly identified.
% The overlapping peak has the counts from both peaks, and the shape will be skewed.
% Assessing the peak shape with numbers is not straightforward, but the FWHM/FWTM can be used to compare the peak shape of different spectra.
% This requires the peaks to be standing alone, without any overlap.
% As this is not always the case, and that the metric is not too useful, it is not used in this work.
% However, an analyst should be aware of the peak shape, as it can be useful if a skewed peak is found and lacking a clear explanation.

% Overlapping peaks, ASTM 1508:
% "9.2.5 Next look for peak overlaps. If (1) a peak is displaced
% relative to its marker, (2) relative intensities are wrong, or (3)
% there is a distortion in the peak shape, then there is an
% overlapping element. All three conditions will be present in an
% overlap situation."


% \subsubsection*{Shadowing}
% \brynjar{From ISO 15632: "In many cases the specific geometry of the EDS detector at a particular SEM/EPMA chamber can result in a reduction of the net active sensor area as expected after subtraction of shadowing area caused by the window grid. For example, a falsely mounted collimator, electron trap or shadowing by other parts in the SEM chamber can reduce additionally the illumination of the detector with X-rays. A practical procedure how to determine experimentally the effective area of an EDS detector and under which conditions are described in Reference [5]."}
% % [5] Procop M., Hodoroaba V.-D., Terborg R., Berger D., Determination of the Effective
% % Detector Area of an Energy-Dispersive X-Ray Spectrometer at the Scanning Electron
% % Microscope Using Experimental and Theoretical X-Ray Emission Yields, Microsc. Microanal.,
% % 22 (2016), pp. 1360-1368
% % Shadowing



% \subsubsection*{Energy dependence of instrumental detection efficiency - K/L ratio}
% \brynjar{ISO 15632: "The minimum specification for the energy dependence of the instrumental detection efficiency shall be
%     the intensity ratio of a low energy line and a high energy line in the characteristic X-ray spectrum of a
%     given material. This ratio shall be given as the net peak area of the L series lines divided by the net peak
%     area of K$\alpha$ series lines in the spectrum of a pure nickel or copper specimen, excited by a 20 keV electron
%     beam perpendicular to the specimen surface and collected by the detector at a take-off angle of 35 degrees. The
%     specimens to be used, the measurement conditions, the calculation of L/K ratio and its conversion for
%     TOA not equal 35 degrees are given in Annex B. \dots These measures are only appropriate for a detector thick enough to absorb at least 95 % of the incident
%     X-ray energy at 8 keV."}


% \subsubsection*{Optimal WD}
% \url{https://www.cstl.nist.gov/div837/837.02/epq/dtsa2/FaultsFoiblesEDS.pdf}
% % \brynjar{En geometrigreie. Er det forskjellig for ulike materialer? Passer under acquisition params}