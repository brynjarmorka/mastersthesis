\chapter{Conclusions}
\label{ch:conclusion}
% Er det viktigste, etter abstract

This chapter gives the conclusions from the work in this thesis. 
First all the sub-conclusions from the discussion are given, and then the main conclusion of this work is given.
The aim of this work, to improve SEM EDS analysis, is put into perspective.
Suggestions for future work are given in \cref{ch:future_work}.

% TODO: Magnus "making the notebook avaliable is a contribution. post the notebook on a old issue"
% TODO: put my notebook in: https://github.com/hyperspy/hyperspy/issues/2332

% conc: si at 50:50 er riktig er farlig, fordi flere feil kan føre til riktig resultat når jeg bare er ute etter et tall.


% TODO: Why is the metrics work important? Magnus: "most people do not think about the stuff i work with. So it is important (to know your equipment)" (also add in discussion)


% Til konklusjon
% 30 kV er bra. Er det best?
% Manualen og opplæring anbefaler disse parameterne [...], men jeg anbefaler disse [...] som gir 20% mer counts, få strays or gode peaks og bra kvantifisering.
% Eventuelt, detektoren var satt opp godt nok, så det var ikke store forbedringspotensialer.



% Sub-conclusions

% the acquisition parameters

% voltage series
The two voltage series acquired yielded poor spectra for the 5 kV and 10 kV spectra, due to few counts, because the beam current was constant.
The lower quality spectra were useful to see how the selected performance parameters varied with quality.
Acquiring better voltage series can probably be done with a constant ICR, i.e. compensating for lower $E_0$ with higher $i_b$.
Another possible solution is to keep $i_b$ constant, but use a lower PT to get sufficient amount of counts at all voltages.

% Dead time
DT is traditionally used as a implicator for the occurances of coincidence events, but this is outdated as it is based on Si(Li) detectors.
The usefulness of DT is limited, as it only gives information about total acquisition time.
% In this work, the possible use of DT is only to figure out how long the acquisition of a spectrum will be. 

% suitable levels of coincidence events
Finding acquisition settings which give a suitable low level of coincidence events must be done with a test spectrum.
In general, coincidence events are first present at very high ICRs.
Most of the spectra acquired in this work had low levels of coincidence events.
This is probably due to the relatively low ICRs used, compared to the capability of the detector.
Additionally, the software used, AZtec, may have reduced the amount of coincidence events by data manipulation, but as the software is poorly documented, this is not confirmed.



% getting good fit
Achieving good model fit is important in EDS analysis, and the fit is dependent on the inputs, which are the elements, $E_0$, and the data points in the spectrum.
The elements must be correct, and thus a through qualitative analysis is necessary.
Mislabeling of elements in a qualitative analysis is less probable with more separated peaks, i.e. with better energy resolution, which is achieved with higher PT.
When the elements are set correct, a computer can seperate peaks better than the human eye, and thus in the analysis spectrum the PT can be lowered in favour of a higher throughput.
The data points used in the model fit should not include the noise peak, and preferably not low energy peaks if they are not of interest.
Slicing the spectrum at the effective beam energy will also improve the fit slightly, and prevents extrapolated values above $E_0$ if the model later is plotted.


% sub-conclusions for the detector

% stray artifacts
The collimator on the EDS detector used is probably working as intended, based on that the spectra have low amount of stray radiations.
However, there are carbon contaminations present in the chamber or on the specimen analysed.
Post acquisition BSE images show darker areas on the specimen, which are probably due to carbon deposition after beam exposure.
In addition to the C peak, the GaSb spectra show a low peak which match with the O K$\alpha$ peak.
Regarding the metrics and quantification, both the peak from C and O are below the lines of interest in this work, and thus they were not further investigated.

% Sb M lines
Another low energy line observed in the GaSb spectra is the Sb M$\zeta$ peak.
No Sb M-lines are tabulated in the HyperSpy liberary or the X-ray databooklet, but the line is clearly visible in the GaSb spectra, and not in the GaAs spectra.
The line was identified as Sb M$\zeta$ through AZtec, the plot of $\mu_\rho$, and Y. Liao \cite{liao2006practical}.
A signal in the GaSb spectra was also observed from what is probably due to Sb M$\gamma$, but this signal was weak.

% Calibration of the detector
The EDS detector used, an Oxford Instruments X-max$^N$ detector, was confirmed to be well calibrated, at least for peaks above $1$ keV.
The spectra recorded had scale and offset which match the settings used in the acquisition, and the recorded peak deviations were deviating at most $\pm 2$ eV from the expected peak positions.
The lower energy peaks of C and O were not investigated in the calculations, but showed a bigger deviation from the expected peak positions.
This is most likely due to the calibration routine of the detector, which is aimed at peaks from around 1 keV and above.
If the lower energy peaks are to be analysed, the spectra acquired should be calibrated for the lower energy peaks.


% Use of peak ratios
The peak ratios recorded, which show a difference between Ga K$\alpha$/L$\alpha$ in GaAs and GaSb, indicate relative absorption differences between the two materials.
Peak ratios can be used to assess stray ratiations, but this requires special specimen geometries which are not used in this work.

% Use of the duane-hunt limit
The Duane-Hunt limit can be used to check the metadata value of $E_0$, and to verify the conductive path between the specimen and ground, which both are important for quantification.
The metric does not need much interpretation, as it is a simple pass/fail test.
If the deviation is too high, e.g. $>1$ keV, the specimen and setup must be checked for errors.
All the Duane-Hunt limits calculated in this work were well within the accepted range, with an average deviation of $0.06$ keV.

% The Fiori metric
The Fiori P/B ratio is useful to assess the quality of an acquired spectrum from the selected settings, but the use of the metric to compare different detectors is limited.
The Fiori metric is highly dependent on the specimen (GaAs $>$ GaSb, and pure $>$ mixture), and thus comparing different detectors requires that the same (standard) specimen is used.
If the Fiori metric is used to compare different detectors, it is also important to calculate the metric the same way, and using model fitting is both recommended and the closest solution to the original definition of the metric.
As the metric is dependent on the acquisition settings, it could probably be used to find the optimal settings for a given specimen.
When a spectrum is recorded, and the counts in the background and peaks increase equally much, the Fiori metric will be constant.
Using this fact, the metric can be used to find out where longer acquisition time or higher ICR would not increase the SNR, i.e. having a constant quality.
In other words, finding the acquisition settings which give enough counts at a quality limit.
This is useful to acquire higher resolution, bigger areas, or more spectra in the same time, or to reduce the acquisition time on expensive equipment.


% summery of the detector
The EDS detector used is performing well.
The energy resolution at $127 \pm6$ eV is as the specifications, and the energy axis is calibrated well.
The Duane-Hunt limits are correct.
The highest Fiori P/B achieved was 770 for GaAs, and 410 for GaSb.

% the acquisition parameters
Achieving good spectra requires enough counts in the peaks, which is affected by $E_0$, $i_b$, and PT.
Overvoltage of $1.4$ gave issues, and the recommendation to use at least $1.5$ might be too low.
Variation in PT affect the energy resolution and the OCR, i.e. the throughput.
A small reduction in energy resolution to greatly increase the throughput is in general recommended, i.e. using a medium to high PT, but not the highest.





% Quantification
% the 50:50 goal 
Assessing the quality of different quantification results based on a reference goal at $50:50$, which is known to be the composition in the analysed specimens, is not an ideal solution.
Two errors can cancel each other out.
However, doing this is gives implications for how the corrections are weigthed, and is regarded as sufficient for this work, when used with caution.

% the need for corrections
The analysis and metrics recorded in this work showed that the GaAs spectrum needed absorption corrections, while the GaSb spectrum needed Z corrections.
In other words, the need for corrections is dependent on the specimen.
Achieving high quality quantification at low $E_0$ requires caution and tailoring, as the possibility of severe overvoltage issues is high.

% AZtec
When quantifying SEM EDS spectra, AZtec can give accurate results, but the software is a black box and lacks proper user-end documentation.
The results from the uncorrected and bulk corrected quantification routines implemented in the code in this project could not match the results from AZtec.
The quantification in AZtec is using XPP bulk corrections, which can correct for both GaAs and GaSb, if implemented correctly.

% TEM routine
The quantification of SEM EDS data with the thin film approximation is too unreliable, but gives two implications.
Firstly, the routines for quantification of thin and bulk specimens are very similar, but not directly transferable, and should be avoided.
Secondly, the CL results from HyperSpy show that the GaAs specimen is quantified more accurately with absorption corrections, while the GaSb specimen needs other corrections.

% ZAF A
The ZAF absorption correction routine implemented in this work is most useful as an indicator of possible improved results with absorption corrections.
The assumption that the Z effects in a specimen cancel out is observed to not be true for GaSb, but is probably true for GaAs.
The implementation used show that the method could provide reasonable results, but that the specific implementation is too simple.

% the XPP corrections
The XPP corrections implemented could both be used for the GaAs and GaSb specimen, as the model can be used to give different corrections based on the needs of the specimen.
The GaAs specimen was best corrected with $f(\chi)$, i.e. XPP absorption corrections, where the average deviation from $50:50$ was $5$ at.\%.
The GaSb specimen was best corrected with $F$, i.e. XPP Z corrections, where the average deviation from $50:50$ was $6$ at.\%.
Plots of the implemented XPP model show that there are some differences to the literature, but the model is close enough to be used.
The differences are probably both due to the implementation, which is not guaranteed to be bug free, and the input parameters, which can change the output of the model drastically.
One input parameter with major effects is the mass absorption coefficients ($\mu_\rho$), which needs refinement.
Values for $\mu_\rho$ were found through HyperSpy, which are less accurate at the relevant energies ($0.03-3$ keV) and around absorption edges, and these values are not quite equal to the values given in the paper that presents the XPP model.


% a complete quantification routine
A complete quantification routine for SEM EDS data must give different corrections based on the specimen.
This is most likely done in AZtec, where Oxford Instruments have had many years to refine the equations and input parameters of their model.
The XPP model implemented in this work is a good start, but it is not a finished model. 
It needs refinement for the input parameters and equations, squashing of bugs, and extensive testing.
The results show that the model can work, but is not perfect.


% And main conclusion of this work.

The work in this thesis is split into two parts, where both are aimed at improving the performance of SEM EDS analysis.
The first part illustrate how an analyst can assess the performance of a detector and selected acquisition settings.
This is important to achieve good spectra, as it is always important to "know your equipment".
The metrics tested are suited for SEM EDS analysis, and can be developed further into a more complete test routine, and the theory chapter can be used as a reference for the metrics.
The second part of this work is aimed at improving the quantification of SEM EDS data, with open-soruce bulk corrections in Python.
The results show that the XPP correction model is a good start, but needs refinement.
The implementation of the model is not perfect, but it is a step towards better EDS analysis through open science.
