\chapter{Method}
\label{ch:method}

\ton{I am not sure what more to write in methods, but it does not feel finished. Any tips?}

This chapter describes the method used in this project.
The method is divided into four parts: the materials studied, the instruments used, the acquisition settings in the measurement series, and the data treatment.
The data treatment is described in detail with the documentation and code in the Jupyter notebooks \verb|\cref{Appendix A}|.
The notebooks and the data is avaliable at GitHub, \verb|\cref or cite or footnote or hyperref|.

% from masteravtalen:
% •	Characterize detector and beam parameters and use them for better quantification.
% •	Develop detector characterization and quantification routines for bulk samples in open-source.









\section{Materials and specimen}
\label{method:materials}

In this study, two different materials were studied: GaAs and GaSb.
A TEM Cu grid was added to the specimen to measure stray radiation, but this did not work out as planned, and these measurements are not included in this study.
% ref to appendix:SE_images as Appendix B
SE images of the analyzed areas are included the \hyperref[appendix:SE_images]{Appendix B}.
Overview images are shown in \cref{fig:SE_images:Overview_GaAs_GaSb}, and the close up images are shown in \cref{fig:SE_images:GaAs}, \cref{fig:SE_images:GaSb}, and \cref{fig:SE_images:GaSb_map}.
The analyzed areas are annotated in the images.
Both specimen were assumed to be heterostructures, with a 1:1 ratio of Ga and As or Sb, respectively.


\subsection{GaAs}
\label{method:materials:gaas}

The GaAs specimen was a 300 \textmu m thick GaAs wafer, with a 1:1 ratio of Ga and As.
Ga has Z=31 and As has Z=33, where both have L-peaks in the low energy range, and K-peaks in the mid/high energy range.
See \cref{tab:theory:lineEnergies} for the theoretical line energies.

\ton{What more to write? Should I add a theoretical spectrum from GaAs? Should the SE images be here?}


\subsection{GaSb}
\label{method:materials:gasb}

The GaSb specimen was a 300 \textmu m thick GaSb wafer, with a 1:1 ratio of Ga and Sb.
Ga has Z=31 and Sb has Z=51.
Ga have L-peaks in the low energy range, and K-peaks in the mid/high energy range.
Sb have L-peaks in the mid energy range, and K-peaks in the highest energy range for SEM.
See \cref{tab:theory:lineEnergies} for the theoretical line energies.







\section{Instruments}
\label{method:instruments}

The acquisition of the spectra was done with a Oxford Xmax$^N$ 80 mm$^2$ EDS detector, equipped on a SEM Apreo from FEI.
The Apreo is a field emmission SEM, which in addition to the EDS sensor has a Everhart-Thornley detector, in-lens detectors for SE and BSE, a directional BSE detector, and a navigation camera.
The maximum accelerating voltage is 30 kV, and the maximum beam current is 400 nA.
During this work, the acceleration voltage has been in the range 5-30 kV, and the beam current has been in the range 25-400 pA.
The acquisition settings used in this study are listed in \cref{tab:method:acquisition_settings:voltage} and \cref{tab:method:acquisition_settings:other}.
The instrument is located at the NTNU NanoLab, a NorFab facility.
\brynjar{Cite NorFab?}

\brynjar{IT IS XMAX-N! FIX}
The details about the EDS detector is from \cite{oxford_xmax_80} and the description provided by NanoLab.
The Oxford Xmax$^N$ SDD has a 80 mm$^2$ active area, which gives a solid angle of 0.03409 sr at WD=10 mm.
The detector is from before 2010, with a claimed count rate of > 500,000 cps and throughout > 200,000 cps \cite{oxford_xmax_80}.
The specifications lists an energy resolution of 127 eV, with typical FWHM(Mn K$\alpha$) of 125 eV.
The specified resolution is in compliance with the ISO 15632:2002 standard, the older verison of \cite{iso_qc_15632}.
All elements from Be to Pu can be analyzed.
The detector is mounted at 35\textdegree.
The process time settings range from 1 to 6, and they have no further explanation of what the number indicates.
The detector is operated with the AZtec software, provided by Oxford Instruments NanoAnalysis \cite{aztec_manual}.



\section{Acquisition settings in the measurement series}
\label{method:acquisition_settings}

The settings selected are based on Goldstein \cite{goldstein_scanning_2018} and the parameters which was going to be analyzed.
The acquisition is divided into two parts: acquisition with varying acceleration voltage and constant beam current, and other acquisition settings.
The voltage series is used to study the effect of the acceleration voltage, to figure out if an optimal voltage exists.
The other acquisition settings tested variations of different settings on the GaSb specimen, to see if any of these settings could improve the quality of the spectra.
The varied settings were beam current, process time (PT), and data type.
The acquisition settings are listed in \cref{tab:method:acquisition_settings:voltage} and \cref{tab:method:acquisition_settings:other}.
An overview of all acquisition settings for each spectrum is given in \cref{tab:method:acquisition_settings:all_spectra}.
Dead time (DT) and input count rate (ICR) is a function of the other settings.


% constant variables
Some variables were kept constant during allt the acquisition.
The tilt of the specimen was kept at 0\textdegree.
The live time for each spectrum was 120 s.
The range of the spectra was 0-20 keV, with a scale of 10 eV/channel.


\begin{table}[phtb]
    \begin{center}
        \caption{
            Measurement series A and B, which tested different $E_0$ on GaAs and GaSb.
            All spectra were recorded with 0-20 keV energy range and 2048 channels at working distance 10 mm (optimal for the detector), and the at TOA 35\textdegree.
            %
        }
        \renewcommand*{\arraystretch}{1.2}
        \label{tab:method:acquisition_settings:voltage}
        \begin{tabular}{p{2cm}p{3cm}p{8.6cm}}
            \hline
            \textbf{Setting}    & \textbf{Values}      & \textbf{Comment}                                                                                                                                                                                                                                               \\
            \hline
            Beam energy         & 30, 15, 10, and 5 kV & To study the effect of different $E_0$.                                                                                                                                                                                                                        \\
            % Beam current        & 25 pA (GaAs), 50 pA (GaSb) & Set to get sufficient amount of counts and maximum DT around 30\% at 30 kV. Then kept constant when $E_0$ was decreased in the voltage series. Different beam current for the specimen was chosen to see the effects on Ga. \\
            Beam current        & 25 pA or 50 pA       & 25 pA for GaAs and 50 pA for GaSb. Set to get sufficient amount of counts and maximum DT around 30\% at 30 kV. Then kept constant when $E_0$ was decreased in the voltage series. Different beam current for the specimen was chosen to see the effects on Ga. \\
            Process time        & 6 (maximum)          & Set to the highest to influence the acquisition as little as possible, i.e. giving the highest energy resolution.                                                                                                                                              \\
            DT GaAs             & 25\%, 12\%, 6\%, 3\% & DT is not a setting, but a result of other settings. The low DT at the lower $E_0$ was due to fewer counts.                                                                                                                                                    \\
            DT GaSb             & 44\%, 18\%, 9\%, 4\% &                                                                                                                                                                                                                                                                \\
            Data type           & PointID              & The voltage series was only done with PointID, while mapping was tested in the measurement series summarized in \cref{tab:method:acquisition_settings:other}.                                                                                                  \\
            Live time           & 120 s                & To get sufficient amount of counts.                                                                                                                                                                                                                            \\
            Additional spectrum & GaAs, 30 kV, 50 pA   & An additional spectra on GaAs with the same settings at the 30 kV spectrum of GaSb. Taken to directly compare the 30 kV spectra.                                                                                                                               \\

            \hline
        \end{tabular}
    \end{center}
\end{table}
%%%%%%%%%%%%%%%%%%%%%%%%%%%%%%%%%%%%%%%%%%%%%%%%%%%%%%%%%%%%%%%%%%%%%%%%%%%%%%%%%%%%%%%%
%%%%%%%%%%%%%%%%%%%%%%%%%%%%%%%%%%%%%%%%%%%%%%%%%%%%%%%%%%%%%%%%%%%%%%%%%%%%%%%%%%%%%%%%
%%%%%%%%%%%%%%%%%%%%%%%%%%%%%%%%%%%%%%%%%%%%%%%%%%%%%%%%%%%%%%%%%%%%%%%%%%%%%%%%%%%%%%%%
\begin{table}[phtb]
    \begin{center}
        \caption{
            Measurement series B and C, which tested variation in $i_b$, PT, and $E_0$.
            These measurements were only done on the GaSb specimen.
            Data type, live time, energy range, channels, WD, and TOA as in \cref{tab:method:acquisition_settings:voltage}.
        }
        \renewcommand*{\arraystretch}{1.2}
        \label{tab:method:acquisition_settings:other}
        \begin{tabular}{p{2cm}p{3cm}p{8.6cm}}
            \hline
            \textbf{Setting} & \textbf{Values}     & \textbf{Comment}                                                                               \\
            \hline
            Beam energy      & 15 and 30 kV        & Further inspect the effect of the Sb K-peaks just below 30 kV.                                 \\
            Beam current     & 50, 200, and 400 pA & To study ICR and DTs, with values given in \cref{tab:method:acquisition_settings:all_spectra}. \\
            Process time     & 1, 2, 4, and 6      & To study the effect of PT, e.g. on energy resolution and coincidence events.                   \\
            % Data type        & PointID and Map     & Compare results from a point and a map, with eventual denoising in the map.                    \\
            %  &                     &                                                                                                       \\
            %  &                     &                                                                                                       \\
            %  &                     &                                                                                                       \\
            \hline
        \end{tabular}
    \end{center}
\end{table}



\begin{table}[phtb]
    \begin{center}
        \caption{
            Experimental settings of acquired and used spectra in this work.
            % \brynjar{Comment from Ton: Add a column with spectra names, e.g. A, B, C, D.}
            The spectra are divided into five groups: (A) voltage series GaAs, (B) voltage series GaSb, (C) different PT, (D) different $i_b$, $E_0$, and PT, and (E) maps.
            Parameters not specified are as in \cref{tab:method:acquisition_settings:voltage}.
        }
        \renewcommand*{\arraystretch}{1.2}
        \label{tab:method:acquisition_settings:all_spectra}
        \begin{tabular}{lllllllp{3.5cm}}
            \hline
            \textbf{Group}                     & \textbf{Sample} & \textbf{$E_0$} & \textbf{$i_b$} & \textbf{PT} & \textbf{ICR} & \textbf{DT} & \textbf{Comment}    \\
                                               &                 & [kV]           & [pA]           &             & [k counts]   & [\%]        &                     \\
            \hline
            \emph{GaAs $E_0$ series}           &                 &                &                &             &              &             &                     \\
            A                                  & GaAs            & 5              & 25             & 6           & 0.88         & 3           &                     \\
            A                                  & GaAs            & 10             & 25             & 6           & 1.75         & 6           &                     \\
            A                                  & GaAs            & 15             & 25             & 6           & 3.3          & 12          &                     \\
            A                                  & GaAs            & 30             & 25             & 6           & 8            & 25          &                     \\
            A                                  & GaAs            & 30             & 50             & 6           & 16.4         & 44          & Additional spectrum \\
            \hline
            \emph{GaSb $E_0$ series  }         &                 &                &                &             &              &             &                     \\
            B                                  & GaSb            & 5              & 50             & 6           & 1.08         & 4           &                     \\
            B                                  & GaSb            & 10             & 50             & 6           & 2.3          & 9           &                     \\
            B                                  & GaSb            & 15             & 50             & 6           & 5.7          & 18          &                     \\
            B                                  & GaSb            & 30             & 50             & 6           & 17           & 44          &                     \\
            \hline
            \emph{PT change}                   &                 &                &                &             &              &             &                     \\
            C                                  & GaSb            & 30             & 50             & 4           & 17           & 13          &                     \\
            C                                  & GaSb            & 30             & 50             & 2           & 17           & 7           &                     \\
            C                                  & GaSb            & 30             & 50             & 1           & 17           & 4           &                     \\
            \hline
            \emph{$i_b$, $E_0$, and PT change} &                 &                &                &             &              &             &                     \\
            D                                  & GaSb            & 30             & 400            & 1           & 160          & 28          & Very high counts    \\
            D                                  & GaSb            & 15             & 200            & 6           & 22           & 53          & 61 s live time      \\
            D                                  & GaSb            & 15             & 400            & 6           & 42           & 77          & 59 s live time      \\
            %             & GaSb            & 15             & 400             & 3           & 50           & 25          & Scratched area      \\
            %             & GaSb            & 30             & 25              & 6           & 6.5          & 20          & Scratched area      \\
            % \hline
            % \emph{Maps}                &                 &                &                &             &              &             &                     \\
            % E                          & GaSb            & 15             & 400            & 3           & 43           & 22          & Map 1               \\
            % E                          & GaSb            & 15             & 400            & 3           & 43           & 22          & Map 2               \\
            \hline
        \end{tabular}
    \end{center}
\end{table}

































\section{Data treatment}
\label{method:data_treatment}

The data treatment is divided into three parts: acquisition and extraction, performance parameters, and quantification.

\subsection{Acquisition and extraction - AZtec}
\label{method:data_treatment:aztec}

Acquisition and extraction is done with AZtec, provided by Oxford Instruments NanoAnalysis \cite{aztec_manual}.
Steps for acquisition is described in Appendix A of Lundeby \cite{lundeby_improving_2019}.
Point measurements was done with "Point \& ID", and map measurements was done with "Map".
The point spectra was extracted as ".emsa" files.
The data from the map spectra was exported as ".emsa", ".raw", ".rpl", and ".txt" files, which was merged to one ".hdf5" file with the script from Lundeby in A.3 \cite{lundeby_improving_2019}.
The merge script and the relevant Appendix from Lundeby is avaliable at the author of this thesis' GitHub \verb|\cref{} or link|.


Initial qualitative analysis was done with the AZtec software, to get an overview of the spectra.
Additionally, all spectra was analysed with the AZtec software, to get the quantification results and the theoretically calculated k-factors.



\subsection{Performance parameters - Jupyter notebooks}
\label{method:data_treatment:notebook}

The Jupyter notebooks developed runs on Python 3.10, with the packages listed in \cref{tab:method:packages}.
The notebooks are both attatched in the appendix of this thesis, and avaliable at the author of this thesis' GitHub \verb|\cref{} or link|.
Two notebooks were made for the performance parameters of the SEM EDS system: one for analysis of a single spectrum with a ".csv" output, and one for statistics on a set of outputted ".csv" files.
The following paragraphs describes the steps in the notebooks.

\begin{table}[hbtp]
    \begin{center}
        \caption{
            The Python packages.
            All dependencies was installed using the package manager Mamba 1.4.0, with Conda 22.11.1.
            When contributing to HyperSpy, the developer version of HyperSpy was installed using pip 23.0.
        }
        \renewcommand*{\arraystretch}{1.2}
        \label{tab:method:packages}
        \begin{tabular}{p{4cm}p{10.6cm}}
            \hline
            \textbf{Package name} & \textbf{Description}                                          \\
            \hline
            HyperSpy 1.7.4        & Multi-dimensional data analysis toolbox.                      \\
            Plotly 5.13.1         & Interactive, open-source, and browser-based graphing library. \\
            Numpy 1.23.5          & Scientific computing with Python.                             \\
            Pandas 1.5.3          & Data analysis package.                                        \\
            JupyterLab 3.6.2      & Web-based interactive development environment.                \\
            SciPy 1.10.1          & Scientific computing with Python.                             \\
            Matplotlib 3.7.1      & Static, animated, and interactive visualizations.             \\
            \hline
        \end{tabular}
    \end{center}
\end{table}


\subsubsection{Performance parameter notebook}
\label{method:data_treatment:notebook:performance}

The performance parameter notebook does the following steps:

\begin{enumerate}
    \item Import the packages
    \item Import the data with HyperSpy, and set the elements in the spectrum
    \item Set the variables like the beam current, acceleration voltage, ICR, and PT for the output
    \item Slice off the noise peak
    \item Calculate the Duane-Hunt limit, and slice the spectrum to the limit
    \item Make a model of the spectrum, with the background as a polynomial and the peaks as Gaussian functions
    \item Fit the background, then fit the whole model
    \item Calibrate the offset and scale
    \item Calibrate the energy resolution
    \item Calibrate the energy and width of the peaks
    \item Calculate Fiori P/B, peak intensities, FWHMs, and peak deviations
    \item Calculate the relavant peak ratios
          % \item Calculate the relavant FWTM/FWHM 
    \item Calculate the total count statistics
    \item Make a DataFrame with the results and output it as a ".csv" file
\end{enumerate}


The statistics notebook loads the wanted ".csv" files into DataFrames, which are then used to calculate the statistics and look at differences in the metrics.
Some of the data is plotted, some are shown in tables.


























\subsection{Quantification - HyperSpy and Jupyter notebooks}
\label{method:data_treatment:quantification}


Intro \dots



