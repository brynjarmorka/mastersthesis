\chapter{Future work}
\label{ch:future_work}

% future works: had these aims, but on this and this the work is too short. Are posibilities like a, b, c
% future works: maps - ikke legge til nå, viktigere å runde av. Kan være i future works. alternativ til point er maps (med denoising i HS, kan ha med noen resultater og figurer i future works)
% future works: factorless
% future works: det jeg har prøvd er første iterasjon i loopen. ved ukjent comp må corrections kjøre flere runder, evt med forskjellig utgangspunkt.


Repeat suggestions from the discussion.

Briefly introduce other possible test.
Linearity and stability, shadowing, optimal WD.


% \brynjar{TODO from Ton: "if not used, do not include them. But consider to use them in discussion leading to concrete future work (last chapter)"}

% Other test are described in the literature, but they were not used in this work.

% \subsubsection*{Linearity and stability}
% In Goldstein \cite[p. 232]{goldstein_scanning_2018} it is stated that the two most important tests for an EDS detector are linearity and stability.
% This is stated in the section about what to look for when buying a detector.
% The reason for not including these two test in the work is mainly their dependence on a way to measure the probe current, which e.g. can be done with a Faraday cup.
% Such tests could be used in the future and are therefore briefly described here.
% Linearity of the detector is that the number of X-rays measured is proportional to the number of X-rays generated.
% Stability is that the detector resolution and the peak positions does not change significantly with different probe currents.
% Both tests require multiple spectra of the same sample with different probe currents.
% The ISO 22309 standard on quantification of EDS spectra \cite{iso_quantification_22309} also mentions these two tests.

% % \brynjar{ISO 22309: "Whatever type of detector is being used, the count rate capabilities of the system should be checked by
% %     comparing a spectrum obtained at a count rate below 2 000 counts/s with a spectrum obtained at the highest
% %     count rate to be used, to look for peak shift and pile-up distortion that might affect relative peak heights. A
% %     minimum of two checks on beam stability, using a Faraday cup, or a known reference specimen, should be
% %     made prior to and following the analysis."}

% \subsubsection*{Shadowing}
% \brynjar{From ISO 15632: "In many cases the specific geometry of the EDS detector at a particular SEM/EPMA chamber can result in a reduction of the net active sensor area as expected after subtraction of shadowing area caused by the window grid. For example, a falsely mounted collimator, electron trap or shadowing by other parts in the SEM chamber can reduce additionally the illumination of the detector with X-rays. A practical procedure how to determine experimentally the effective area of an EDS detector and under which conditions are described in Reference [5]."}
% % [5] Procop M., Hodoroaba V.-D., Terborg R., Berger D., Determination of the Effective
% % Detector Area of an Energy-Dispersive X-Ray Spectrometer at the Scanning Electron
% % Microscope Using Experimental and Theoretical X-Ray Emission Yields, Microsc. Microanal.,
% % 22 (2016), pp. 1360-1368
% % Shadowing


% \subsubsection*{Optimal WD}
% \url{https://www.cstl.nist.gov/div837/837.02/epq/dtsa2/FaultsFoiblesEDS.pdf}
% % \brynjar{En geometri-greie. Er det forskjellig for ulike materialer? Passer under acquisition params}

Also mention maps?