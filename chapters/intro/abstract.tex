\chapter*{Abstract}
% - Finding performance parameters and implementing bulk corrections for SEM EDS data in open-soruce software

\label{ch:abstract}
\addcontentsline{toc}{chapter}{Abstract}

% what I have done
This work is aimed at improving SEM EDS analysis through ($1$) measurement of performance parameters and ($2$) bulk corrections for qualitative analysis in open-source software.
The developed routines were tested on EDS spectra of GaAs and GaSb wafers, acquired using a FEI Apreo SEM with an Oxford Instruments X-Max$^N$ $80$mm$^2$ EDS detector.

% Part 1 - be concrete. use conclusions
The performance parameters are calculated in a Jupyter notebook to test the EDS setup and the selected acquisition settings, with the goal of revealing potential errors and finding the optimal settings.
The parameters investigated are energy resolution, Fiori peak-to-background ratio (P/B), Duane-Hunt limit, peak ratios, scale, offset, and peak deviations.
The results from the equipment used in this work show that the system specifications are met.
The energy resolution was measured to be on average $127$ eV, with some variations between specimen, reference peak, and acquisition settings.
The highest Fiori P/B ratio was $770$ for Ga K$\alpha$ in GaAs, and $410$ for Sb L$\alpha$ in GaSb.
The scale was $10$ eV/channel, confirming that the energy axis was well calibrated.
Low levels of coincidence counts were observed when the input count rate was below $50$k cps.
Settings and factors affecting the performance parameters are identified, for example how the Fiori P/B is observed to be highly specimen dependent.
It is discussed how different calculation methods of energy resolution and Fiori P/B affects the results, making the metrics less suited to compare different EDS hardware.
In future work, the performance test routine should include metrics like the detectors' linearity, stability, and efficiency, which all require an expanded notebook and extended acquisition routine.


% part 2 - what have i done, which is new
To improve quantification routines in SEM EDS analysis, implementation of bulk corrections in open-source Python software is explored.
Bulk corrections from the model made by Pouchou and Pichoir in 1991 are implemented, more specifically the XPP version.
The results are compared to the commercial software AZtec, to the quantification routine in the open-source Python package HyperSpy, and to a simple implementation ZAF corrections.
The quantification routine in HyperSpy use the thin film approximation, and this study show that SEM bulk specimen should not be treated as thin films.
AZtec is most accurate, but it is a black box software where the user-end documentation is limited.
Black box software is not flexible and not transparent; thus the results cannot be verified, and the analysis cannot be widened.
The implemented XPP model seems more promising than the ZAF model, but this first implementation needs to be further refined, for example through updated input factors like the mass absorption coefficient. % $\mu_\rho$.
% The transparency and flexibility of open-source Python software makes the implemented XPP model a good candidate for further development.
The code in this work is published under the MIT license, and the work is a step towards better EDS analysis. % through open science.



% Keywords: SEM EDS, performance parameters, XPP bulk corrections, quantification, open-source Python code, materials characterization.