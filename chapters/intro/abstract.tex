\chapter*{Abstract}

\label{ch:abstract}
\addcontentsline{toc}{chapter}{Abstract}


This master's thesis presents a comprehensive investigation into improving the quantitative analysis of energy dispersive X-ray spectroscopy (EDS) data acquired in scanning electron microscopy (SEM). 
The study focuses on two key aspects: ($1$) describing and testing performance parameters of the setup and the acquisition settings, and ($2$) implementing XPP bulk corrections.
Both are done using open-source Python code.

The first part of the thesis examines the performance parameters of SEM EDS, delving into the entire process from signal creation, to detection, and to interpretation. 
The performance parameters investigated are directed towards the system performance and of the selected acquisition settings, including the Fiori peak-to-background ratio (P/B), energy resolution, Duane-Hunt limit, peak ratios, scale, offset, and peak deviations. 
Through thorough analysis and optimization of these parameters, the aim is to enhance the accuracy, and reliability of SEM EDS measurements.

In the second part, XPP bulk corrections based on the PAP (Pouchou and Pichoir) model are explored. 
The thesis focuses on implementing the XPP corrections using open-source Python code, allowing for efficient and accurate quantification of EDS data. 
The XPP corrections address sample absorption and depth distribution effects, to improve the accuracy of elemental quantification in SEM EDS analysis.

Throughout the thesis, open-source Python code is utilized to facilitate the analysis of performance parameters and the implementation of XPP bulk corrections. 
Leveraging the flexibility and functionality of Python, the developed code enables researchers to optimize performance parameters and apply advanced bulk corrections, leading to more accurate and reliable quantitative analysis in SEM EDS.

The findings of this study contribute to the broader field of materials characterization by offering insights into the optimization of performance parameters and the application of XPP bulk corrections in SEM EDS. 
The open-source Python code developed in this work provides a valuable resource for the scientific community, promoting transparency, reproducibility, and further advancements in quantitative EDS analysis.

Keywords: SEM EDS, performance parameters, XPP bulk corrections, quantification, open-source Python code, materials characterization.
\brynjar{Mention HyperSpy? Or the fact that the results are not perfect? AZtec?}