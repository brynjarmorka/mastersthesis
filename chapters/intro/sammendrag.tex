\chapter*{Sammendrag}
\label{ch:Sammendrag}
\addcontentsline{toc}{chapter}{Sammendrag}

Målet med dette arbeidet er å forbedre analysen av data fra energidispersiv røntgenspektroskopi i skanneelektronmikroskop (SEM EDS) gjennom ($1$) måling av ytelsesparametere og ($2$) bulkkorrigeringer for kvalitativ analyse i åpen kildekodeprogramvare i Python.
De utviklede rutinene ble testet på EDS-spektre av GaAs- og GaSb-wafere, hvor dataen ble innhentet fra en FEI Apreo SEM med en Oxford Instruments X-Max$^N$ $80$mm$^2$ EDS-detektor.


Ytelsesparameterne blir beregnet i en Jupyter notebook for å teste EDS-oppsettet og de valgte innhentingsinnstillingene. 
De undersøkte parameterne er energioppløsning, Fiori topp-til-bakgrunn forholdet (P/B), Duane-Hunt-grensen, toppforholdstall, energiskala, forskyvning og toppavvik. 
Resultatene viser at systemspesifikasjonene til det brukte utstyret er oppfylt.
Energioppløsningen ble målt til $127$ eV, med noe variasjon mellom prøvene, referansetoppene og innhentingsinnstillingene. 
Den høyeste Fiori P/B var $770$ for Ga K$\alpha$ i GaAs og $410$ for Sb L$\alpha$ i GaSb. 
Skalaen var $10$ eV/kanal, som viser at energiaksen var godt kalibrert.
Det ble observert lave nivåer av sammenfallende tellinger når telleraten var under $50$k tellinger per sekund. 
Innstillinger og faktorer som påvirker ytelsesparameterne er identifisert, for eksempel hvordan Fiori P/B-forholdet er avhengig av prøven. 
Det blir diskutert hvordan ulike beregningsmetoder for energioppløsning og Fiori P/B påvirker resultatene, noe som begrenser sammenlikningsgrunnlaget mellom ulike systemer.
I fremtidig arbeid bør rutinen for ytelsestesting bli utvidet for blant annet å bestemme detektorenes linearitet, stabilitet og effektivitet, noe som krever en utvidet Jupyter notebook og mer omfattende innhentingsrutiner.


For å forbedre kvantifiseringsrutinene i SEM EDS-analyse, er implementeringen av bulkkorrigeringer i Python utforsket.
Den mest lovende modellen er basert på arbeidet til Pouchou og Pichoir i 1991, spesifikt XPP-versjonen.
Resultatene er sammenlignet med den kommersielle programvaren AZtec, en kvantifiseringsrutine i HyperSpy, og en enkel implementering av ZAF-korrigeringer.
Kvantifiseringsrutinen i HyperSpy bruker tynnfilmtilnærmingen, og denne studien viser at SEM-bulkprøver ikke bør bli behandlet som tynnfilmer i kvantifisering.
AZtec er mest nøyaktig, men det er en svart-boks programvare der dokumentasjonen for brukeren er begrenset, hvilket betyr at brukeren ikke kan verifisere resultatene og ikke utvide analysen etter egne behov.
Den implementerte XPP-modellen ser mer lovende ut enn ZAF-modellen, men denne første implementeringen trenger videre utvikling, for eksempel ved oppdatering av inputvariabler som masseabsorpsjonskoeffisienten. 
Koden i dette arbeidet er publisert under MIT-lisensen, og arbeidet representerer et skritt mot bedre EDS-analyse. 