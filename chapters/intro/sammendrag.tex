\chapter*{Sammendrag}
\label{ch:Sammendrag}
\addcontentsline{toc}{chapter}{Sammendrag}

Dette arbeidet har som mål å forbedre analysen av data fra energidispersiv røntgenspektroskopi i skanneelektronmikroskop (SEM EDS) ved hjelp av ytelsesparametere og bulkkorrigeringer i åpen kildekodeprogramvare. 
De utviklede rutinene ble testet på EDS-spektre av GaAs- og GaSb-wafere, som ble innhentet ved hjelp av et FEI Apreo SEM med en Oxford Instruments X-Max$^N$ $80$mm$^2$ EDS-detektor.


Ytelsesparametrene beregnes i en Python-notebook for å teste EDS-oppsettet og de valgte innhentingsinnstillingene. 
De undersøkte parameterne inkluderer energioppløsning, Fiori topp-til-bakgrunn forholdet (P/B), Duane-Hunt-grensen, toppforholdstall, energiskala, forskyvning og toppavvik. 
Resultatene viser at systemspesifikasjonene til det brukte utstyret oppfylles.
Energioppløsningen ble målt til $127$ eV, med noe variasjon mellom prøvene, referansetoppene og innhentingsinnstillingene. 
Den høyeste Fiori P/B-forholdet var $770$ for Ga K$\alpha$ i GaAs og $410$ for Sb L$\alpha$ i GaSb. 
Skalaen var $10$ eV/kanal.
Det ble observert lave nivåer av sammenfallende tellinger når telleraten var under $50$k cps. 
Innstillinger og faktorer som påvirker ytelsesparametrene er identifisert, for eksempel hvordan Fiori P/B-forholdet er avhengig av prøven. 
Det blir diskutert hvordan ulike beregningsmetoder for energioppløsning og Fiori P/B påvirker resultatene, noe som begrenser sammenlikningsgrunnlaget mellom ulike systemer.
I fremtidig arbeid bør rutinen for ytelsestesting utvides for å bestemme detektorenes linearitet, stabilitet og effektivitet, noe som krever en utvidet notebook og innhentingsrutine.


For å forbedre kvantifiseringsrutinene i SEM EDS-analyse, utforskes implementeringen av bulkkorrigeringer i Python.
Den mest lovende modellen er basert på arbeidet til Pouchou og Pichoir, spesifikt XPP-versjonen.
Resultatene er sammenlignet med den kommersielle programvaren AZtec, en kvantifiseringsrutine i HyperSpy, og en enkel implementering av ZAF-korrigeringer.
Kvantifiseringsrutinen i HyperSpy bruker tynnfilmtilnærmingen, men denne studien viser at SEM-bulkprøver ikke bør behandles som tynnfilmer i kvantifisering.
AZtec er mest nøyaktig, men det er en "black box"-programvare der dokumentasjonen for brukeren er begrenset, hvilket betyr at brukeren ikke kan verifisere resultatene.
% "Black box"-programvare er ikke fleksibel og transparent, noe som resulterer i at resultatene ikke kan verifiseres, og analysen kan ikke utvides.
Den implementerte XPP-modellen ser mer lovende ut enn ZAF-modellen, men denne første implementeringen må forbedres videre, for eksempel gjennom faktorer som inngår i beregningen av massabsorpsjonskoeffisienten. % $\mu_\rho$.
Gjennom åpen kildekodeprogramvaren i Python er den implementerte XPP-modellen et godt utgangspunkt for videre utvikling.
Koden i dette arbeidet er publisert under en MIT-lisens, og dette representerer et skritt mot bedre EDS-analyse gjennom åpen vitenskap.