\begin{table}[phtb]
    \begin{center}
        \caption{
            Variables from the principle of PAP.
            The variables are in the order of appearance.
        }
        %\renewcommand*{\arraystretch}{1.4}
        \label{tab:quantitative:PAP:variables:principle}
        \begin{tabular}{cp{8cm}c}
            \hline
            \textbf{Symbol} & \textbf{Name and/or definition}                                                                                     & \textbf{Units}    \\
            \hline
            $n_A$           & Number of generated primary ionizations from atoms of element A.                                                    & 1 (a number)      \\
            $C_A$           & Mass concentration of element A.                                                                                    & wt\%              \\
            $N^0$           & Avogadro's number. $6.02E23$                                                                                        & 1/mol             \\
            $A$             & Atomic weight of atom A. Do not mix with the parameter $A$ in the parameterization of $\phi(\rho z)$, further down. & Da                \\
            $Q_l^A(E_0)$    & Ionization cross section of element A, with beam energy $E_0$.                                                      & 1 (a probability) \\
            $E_0$           & Beam energy.                                                                                                        & keV               \\
            $F$             & Integral or area of $\phi(\rho z)$                                                                                  & ?                 \\
            $\phi(\rho z)$  & The depth distribution or radiation from a X-ray line.                                                              & ?                 \\
            $\rho z$        & Mass density                                                                                                        & g/cm$^2$          \\
            hline
        \end{tabular}
    \end{center}
\end{table}
