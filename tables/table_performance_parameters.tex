\begin{table}[htp]
    \centering
    \caption{
        Summary of the performance parameters covered in this thesis.
    }
    \renewcommand*{\arraystretch}{1.4}
    \label{tab:eds_performance_parameters}
    \begin{tabular}{p{2.5cm}p{8cm}p{3.5cm}}
        \hline
        \textbf{Parameter}                                                          & \textbf{Definition}                                                                                                                                                                                                                                         & \textbf{Characterizes}                                             \\
        \hline
        \hyperref[theory:eds_performance:duanehunt]{Duane-Hunt limit}               & The effective incident beam energy, i.e. the maximum energy generating X-rays. Found by linear regression \cite{Duane_Hunt_1915,software_dtsaii,goldstein_scanning_2018}. See \cref{fig:duanehunt}.                                                         & Charging issues and verifies the selected $E_0$.                   \\
        \hyperref[theory:eds_performance:energyres]{Energy resolution}              & Measured as the FWHM(Mn K$\alpha$). Can be measured directly or calculated with \cref{eq:estimateFWHM} \cite[Ch. 16.1.1]{goldstein_scanning_2018}. $\textnormal{FWHM}(E) =  \sqrt{2.5 * (E - E_\textnormal{ref}) + \textnormal{FWHM}^2_{\textnormal{ref}}}$ & The detector specifications and the settings used for acquisition. \\
        \hyperref[theory:eds_performance:scaleoffset]{Scale and offset}             & The width of each channel and the zero-offset of the spectrum.                                                                                                                                                                                              & The settings selected by the user.                                 \\
        \hyperref[theory:eds_performance:scaleoffset]{Deviations in peak positions} & How many eV the center of the fitted Gaussian deviate from the theoretical peak center.                                                                                                                                                                     & The accuracy of the calibration.                                   \\
        \hyperref[theory:eds_performance:fiori]{Fiori P/B ratio}                    & A signal-to-noise ratio. The sum of all the counts in a peak divided by the background counts in a $10$ eV window under the peak center  \cite{fiori_peak_background_1982,williams_carter_tem_2009}. See \cref{eq:fiori_pb} and \cref{fig:fiori_pb}.          & The detector quality and quality of the acquired spectrum.         \\
        \hyperref[theory:eds_performance:peakratio]{Peak ratios}                    & Counts in one peak divided by the counts in another peak. With specific specimen geometry, this can be used to characterize strays. \cite{egerton_nio_characterization_1994,ted_pella_nio_tem_2019}.                                                        & Information about stray radiation, and carbon contamination.       \\
        % \hyperref[theory:eds_performance:background_portion]{Background portion}      & The percentage of the total counts which are from the background.                                                                                                                                                                                           & The quality of the acquired spectrum.                              \\
        \hline
    \end{tabular}
\end{table}