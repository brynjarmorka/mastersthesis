\begin{table}[htp]
    \centering
    \caption{
        Summary of the performance parameters covered in this thesis.
    }
    \renewcommand*{\arraystretch}{1.4}
    \label{tab:eds_performance_parameters}
    \begin{tabular}{p{2.6cm}p{5cm}p{5cm}}
        \textbf{Parameter}                                                            & \textbf{Definition}                                                                              & \textbf{Use}                                                       \\
        \hline
        \hyperref[theory:eds_performance:duanehunt]{Duane-Hunt limit}                 & The effective incident beam energy, i.e. the maximum energy generating X-rays.                   & Discover charging issues.                                          \\
        \hyperref[theory:eds_performance:energyres]{Energy resolution}                & The FWHM(Mn K$\alpha$).                                                                          & Verify the detector specifications.                                \\
        \hyperref[theory:eds_performance:scaleoffset]{Scale and offset}               & The width of each channel and the zero-offset of the spectrum.                                   & Verify the settings selected by the user.                          \\
        \hyperref[theory:eds_performance:peakpositions]{Deviations in peak positions} & How many eV the center of the fitted Gaussian deviate from the theoretical peak.                 & The accuracy of the electronics.                                   \\
        \hyperref[theory:eds_performance:fiori]{Fiori P/B ratio}                      & A signal-to-noise ratio. Intensity of a peak divided by the background counts in a 10 eV window. & Assess the both the detector and quality of the spectrum acquired. \\
        \hyperref[theory:eds_performance:peakratio]{Peak ratios}                      & Counts in one peak divided by the counts in another peak.                                        & Discover carbon contamination, or stray information.               \\
        \hyperref[theory:eds_performance:bacgrkound_portion]{Background portion}      & The percentage of the total counts which are from the background.                                & Assess the quality of the acquired spectrum.
    \end{tabular}
\end{table}