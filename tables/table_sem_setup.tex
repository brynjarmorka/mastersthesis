\begin{table}[phb]
    \begin{center}
        \caption{
            Definitions for various terms related to scanning electron microscope, \cref{theory:sem}.
        }
        \renewcommand*{\arraystretch}{1.4}
        \label{tab:sem}
        \begin{tabular}{p{4cm}p{10.6cm}}
            \hline
            \textbf{Name}           & \textbf{Definition}                                                                                                                                                                                        \\
            \hline
            Electron gun            & The source of the electron beam.                                                                                                                                                                           \\
            Beam energy, $E_0$      & The voltage applied that accelerates the electrons in the beam.                                                                                                                                            \\
            Beam current, $i_b$     & The current of electrons emitted from the electron gun. This is not the current hitting the specimen (probe current), as the probe current is limited by the apertures and loss in the column and chamber. \\
            Probe size              & The cross-section of the electron beam when it hits the sample.                                                                                                                                            \\
            Electromagnetic lenses  & The lenses which shape and control the electron beam in the column. Condenser lenses, scanning coils, and objective lens.                                                                                  \\
            Aperture                & A physical barrier with a hole in the middle which allows only part of the signal to pass through.                                                                                                         \\
            Interaction volume      & The volume in the sample where the different signals originate from.                                                                                                                                       \\
            Auger electrons         & A signal of electrons emittet from the surface of the sample.                                                                                                                                              \\
            Secondary electrons     & A surface sensitive signal of electrons used to produce topological images.                                                                                                                                \\
            Backscattered electrons & Beam electrons which have had their trajectory reversed through the scattering processes. The BSE signal has Z contrast.                                                                                   \\
            \hline
        \end{tabular}
    \end{center}
\end{table}