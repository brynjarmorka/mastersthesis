\begin{table}[htbp]
    \begin{center}
        \caption{
            Selected maximum ranges of creation for lines in different specimen, based on the Kanaya-Okayama parameterization.
            The path length of the X-rays are $r \cdot \csc(TOA)$.
            The ranges are used to calculate the ZAF absorption correction factors, listed in \cref{tab:results:ZAF_corrections_factors}.
        }
        %\renewcommand*{\arraystretch}{1.4}
        \label{tab:results:ZAF_corrections_range_r}
        \begin{tabular}{rrrrrr}
            \hline
            \textbf{Specimen} & \textbf{Line} & \textbf{$r$ at 30 kV} & \textbf{$r$ at 15 kV} & \textbf{$r$ at 10 kV} & \textbf{$r$ at 5 kV} \\
            \emph{}           & \emph{}       & \emph{[\textmu m]}    & \emph{[\textmu m]}    & \emph{[\textmu m]}    & \emph{[\textmu m]}   \\
            \hline
            GaAs              & Ga La         & 4.13                  & 1.29                  & 0.65                  & 0.19                 \\
                              & As La         & 4.12                  & 1.28                  & 0.64                  & 0.19                 \\
                              & Ga Ka         & 3.56                  & 0.72                  & 0.08                  & -                    \\
            %&As Ka&&&&\\
            \hline
            GaSb              & Ga La         & 4.02                  & 1.25                  & 0.63                  & 0.19                 \\
                              & Sb La         & 3.92                  & 1.15                  & 0.53                  & 0.09                 \\
            %&Ga Ka&&&&\\
            \hline
        \end{tabular}
    \end{center}
\end{table}
