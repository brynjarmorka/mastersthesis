\begin{table}[htbp]
    \begin{center}
        \caption{
            Selected maximum ranges of creation for lines in different specimen, based on the Kanaya-Okayama parameterization.
            The path length of the X-rays are $r \cdot \csc(TOA)$.
            In \cref{tab:results:ZAF_corrections_factors} the $r$ is divided by a range divider to get an approximation for the mean X-ray generation depth.
        }
        %\renewcommand*{\arraystretch}{1.4}
        \label{tab:results:ZAF_corrections_range_r}
        \begin{tabular}{rrrrrr}
            \hline
            \textbf{Specimen} & \textbf{Line} & \textbf{$r$ at 30 kV} & \textbf{$r$ at 15 kV} & \textbf{$r$ at 10 kV} & \textbf{$r$ at 5 kV} \\
            %\emph{}& \emph{}& \emph{}& \emph{}& \emph{}& \emph{}\\
            \hline
            GaAs              & Ga La         & 4.13 um               & 1.29 um               & 0.65 um               & 0.19 um              \\
                              & As La         & 4.12 um               & 1.28 um               & 0.64 um               & 0.19 um              \\
                              & Ga Ka         & 3.56 um               & 0.72 um               & 0.08 um               & -                    \\
            %&As Ka&&&&\\
            \hline
            GaSb              & Ga La         & 4.02 um               & 1.25 um               & 0.63 um               & 0.19 um              \\
                              & Sb La         & 3.92 um               & 1.15 um               & 0.53 um               & 0.09 um              \\
            %&Ga Ka&&&&\\
            \hline
        \end{tabular}
    \end{center}
\end{table}
