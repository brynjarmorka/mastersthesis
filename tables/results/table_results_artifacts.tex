\begin{table}[phtb]
	\begin{center}
		\caption{
			The artifacts present in the spectra.
			See \cref{fig:results:overviewGaSb_withArtifacts,fig:results:GaSb_voltages,fig:results:GaAs_voltages}.
		}
		\renewcommand*{\arraystretch}{1.4}
		\label{tab:results:artifacts}
		\begin{tabular}{p{3.5cm}p{11.1cm}}
			\hline
			\textbf{Artifact}                                & \textbf{Where the articat is present and a comment}                                                                                                                                                                                                                                                                                                                                                                                             \\
			\hline
			Background                                       & All spectra. Increase with higher count rate.                                                                                                                                                                                                                                                                                                                                                                                                   \\
			Absorption edge effect on background             & Most prominent in GaAs. Reduces the background intensity above the Ga L absorption edge. In \cref{fig:results:GaAs_voltages} panel (b) the background drop from around 600 to around 170 counts.                                                                                                                                                                                                                                                \\
			Noise peak                                       & All spectra. Located almost at 0 keV.                                                                                                                                                                                                                                                                                                                                                                                                           \\
			Coincidence peaks                                & Only the spectra with very high count rates. \cref{fig:results:overviewGaSb_withArtifacts} with GaSb taken at 30 kV, 400 pA, and PT1 show coincidence peaks from: (Sb L + Sb L), (Sb L + Ga K), and (Ga K + Ga K).                                                                                                                                                                                                                              \\
			Tailing background noise from coincidence events & Present in spectra taken at 5, 10, and 15 kV. Coincidence events from two arbitrary counts give a tailing background. Exemplified by the green 15 kV line in \cref{fig:results:GaSb_voltages} panel (a), where vertical lines (one count each) are present between 15 and 20 keV.                                                                                                                                                               \\
			Internal fluorescence peak                       & Visible in some spectra. A low signal, barely a peak in some spectra, at Si K$\alpha$. Most prominent in 30 kV spectra and spectra with high count rates.                                                                                                                                                                                                                                                                                       \\
			Si escape peak                                   & Most GaSb spectra show some escape signal from Sb L$\alpha$ at 1.86 keV, labeled in \cref{fig:results:overviewGaSb_withArtifacts} panel (b). The coincidence counts from (Sb L + Sb L) marked as "Coincidence 1" in \cref{fig:results:overviewGaSb_withArtifacts} panel (a) has one peak at 7.2 keV and one at 7.5 keV, where the latter cound be a combination of coincidence events and escape counts from Ga K$\alpha$ (9.25 - 1.74 = 7.51). \\
			Stray C                                          & All spectra show a C K$\alpha$ peak, with some variation in intensity.                                                                                                                                                                                                                                                                                                                                                                          \\
			Stray O                                          & All spectra of GaSb show an O K$\alpha$ peak. The GaAs spectra have much lower, but still present signal at 0.52 keV.                                                                                                                                                                                                                                                                                                                           \\
			\hline
		\end{tabular}
	\end{center}
\end{table}
