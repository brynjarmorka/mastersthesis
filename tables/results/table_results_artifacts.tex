\begin{table}[phtb]
	\begin{center}
		\caption{
			The artifacts present in the spectra.
			See \cref{fig:results:overviewGaSb_withArtifacts,fig:results:GaSb_voltages,fig:results:GaAs_voltages}.
		}
		\renewcommand*{\arraystretch}{1.4}
		\label{tab:results:artifacts}
		\begin{tabular}{p{3.5cm}p{8.5cm}}
			\hline
			\textbf{Artifact}                          & \textbf{Present in and comment}                                                                                                                                                                                    \\
			\hline
			Background                                 & All spectra. Increase with higher count rate.                                                                                                                                                                      \\
			Absorption edge effect on background       & Most prominent in GaAs. Reduces the background intensity above the Ga L absorption edge. In \cref{fig:results:GaAs_voltages} panel (b) the background drop from around 600 to around 170.                          \\
			Noise peak                                 & All spectra. Located almost at 0.                                                                                                                                                                                  \\
			Coincidence peaks                          & Only the spectra with very high count rates. \cref{fig:results:overviewGaSb_withArtifacts} with GaSb taken at 30 kV, 400 pA, and PT1 show coincidence peaks from: (Sb L + Sb L), (Sb L + Ga K), and (Ga K + Ga K). \\
			Tailing background from coincidence events & Spectra taken at 5, 10, and 15 kV. Coincidence events giving a tailing background, e.g. as vertical counts between 15 and 20 keV for the green line (15 kV) in panel (a) in \cref{fig:results:GaSb_voltages}.      \\
			Internal fluorescence peak                 & A low signal, barely a peak in some spectra, at Si K$\alpha$.                                                                                                                                                      \\
			Si escape peak                             & Only a small escape signal from Sb L$\alpha$, at 1.86 keV.                                                                                                                                                         \\
			Stray C                                    & All spectra show a C K$\alpha$ peak.                                                                                                                                                                               \\
			Stray O                                    & All spectra of GaSb show an O K$\alpha$ peak. The GaAs spectra have a very small signal at 0.52 keV, but not a real peak.                                                                                          \\
			\hline
		\end{tabular}
	\end{center}
\end{table}
