\begin{table}[phtb]
    \begin{center}
        \caption{
            Numbers for the XPP corrected compositions listed in \cref{tab:results:XPP_compositions}.
            The XPP corrections are done in two ways.
            The uncorrected intensity ratio results are listed for reference.
            % The table gives the average and the number of deviations below 5 at.\% and above 10 at.\% are counted, for GaAs and GaSb separated.
            $f(\chi)$ the absorption correction, and $F$ is the generated amount of X-rays, i.e. Z corrections.
            $f(\chi)$ is defined in \cref{eq:theory:quantitative:pap:absorption_correction}, and $F$ is defined in \cref{eq:theory:quantitative:pap:general_principle:F}.
        }
        %\renewcommand*{\arraystretch}{1.4}
        \label{tab:results:XPP_compositions_stats}
        \begin{tabular}{rrrrr}
            \hline
            \textbf{Specimen} & \textbf{Number}          & \textbf{Uncorr.} & \textbf{XPP corr. $(I_A/f(\chi))$} & \textbf{XPP corr. $(I_A/F)$} \\
            %
            \hline

            GaAs              & Average deviation, at.\% & 12 at.\%         & 5 at.\%                            & 11 at.\%                     \\
                              & Deviations $>10$ at.\%   & 4                & 0                                  & 3                            \\
                              & Deviations  $<5$  at.\%  & 0                & 2                                  & 0                            \\
            \hline

            GaSb              & Average deviation, at.\% & 6 at.\%          & 38 at.\%                           & 6 at.\%                      \\
                              & Deviations $>10$ at.\%   & 1                & 9                                  & 1                            \\
                              & Deviations  $<5$  at.\%  & 5                & 0                                  & 3                            \\

            \hline
        \end{tabular}
    \end{center}
\end{table}
