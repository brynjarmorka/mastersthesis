\begin{table}[phtb]
    \begin{center}
        \caption{
            A summary of the performance parameter results.
            %
            %
        }
        \renewcommand*{\arraystretch}{1.4}
        \label{tab:results:performance_summary}
        \begin{tabular}{p{2.5cm}p{4cm}p{7cm}}
            \hline
            \textbf{Parameter}                                            & \textbf{Value}                & \textbf{Comment}                                                                                                                                                                                       \\
            \hline%%%%%%
            \hyperref[results:duane_hunt]{Duane-Hunt limit}               & $E_0 \pm$ 0.2 keV             & For the detector. Average deviation was 0.06 $\pm$0.04 keV, but the value specified include the maximum deviations.                                                                                    \\
            \hyperref[results:energy_resolution]{Energy resolution}       & 127 $\pm$6 eV                 & Parameter for the detector, but highly dependent on PT. Recorded at the highest PT on GaAs and GaSb. The variation is from different reference peaks in the calculation and different $E_0$ and $i_b$. \\
            \hyperref[results:scaleoffset]{Scale}                         & 10.007 $\pm$0.038 eV          & For the detector. The scale of the GaAs spectra varied slightly more than the GaSb spectra.                                                                                                            \\
            \hyperref[results:scaleoffset]{Offset}                        & - 0.205 $\pm$0.004 keV        & For the detector.                                                                                                                                                                                      \\
            \hyperref[results:scaleoffset]{Deviations in peak positions}  & 0-2 eV                        & For the detector.                                                                                                                                                                                      \\
            \hyperref[results:fiori]{Fiori P/B ratio}                     & 770 for Ga K$\alpha$ in GaAs  & For the detector, acquisition parameters and specimen dependent. The value is the highest achieved.                                                                                                    \\
            \hyperref[results:fiori]{Fiori P/B ratio}                     & 410 for Ga K$\alpha$ in GaSb  & "                                                                                                                                                                                                      \\
            \hyperref[results:peak_ratios]{Peak ratios}                   & GaSb > GaAs and $\propto E_0$ & Varying with acquisition parameters and specimen.                                                                                                                                                      \\
            \hline%%%%%%
            \hyperref[results:beam_energy_and_beam_current]{Beam energy}  & 15 kV or 30 kV                & Best results were acquired with higher beam energy.                                                                                                                                                    \\
            \hyperref[results:beam_energy_and_beam_current]{Beam current} & Not a specific number         & Dependent on $E_0$, and needs to give enough counts.                                                                                                                                                   \\
            \hyperref[results:process_time]{Process time}                 & Not a number                  & Should be high, but the highest give long acquisition time.                                                                                                                                            \\
            \hline
        \end{tabular}
    \end{center}
\end{table}
