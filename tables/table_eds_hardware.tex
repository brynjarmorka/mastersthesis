\begin{table}[pht]
    \begin{center}
        \caption{Definitions and references for various terms related to EDS detectors, \cref{theory:eds:hardware}.}
        \renewcommand*{\arraystretch}{1.4}
        \label{tab:eds:hardware}
        % \begin{tabular}{|l|p{10cm}|l|}
        \begin{tabular}{lp{10cm}l}

            \hline
            \textbf{Name}          & \textbf{Definition}                                                                                                                                                                                                    & \textbf{Reference} \\
            \hline
            SDD                    & Silicon drift detector, the most common detector. High count rates, fast processing, and low capacitance.                                                                                                              &                    \\
            Si(Li) detector        & The older type of detectors. Much of the literature published is on these detectors.                                                                                                                                   &                    \\
            Channel                & A discrete energy interval in the measured histogram. Usually 10 eV.                                                                                                                                                   & ISO15632           \\
            Scale                  & Energy per channel.                                                                                                                                                                                                    &                    \\
            Signal intensity       & The number of counts stored in a channel. Signal intensity can also be expressed as counts per second, but counts is used in this work.                                                                                & ISO15632           \\
            Active layer           & The part of the detector where the detectable electron-hole pairs are generated.                                                                                                                                       &                    \\
            Dead layer             & The Si layer in the detector where generated electron-hole pair are not drifted to the anode and thus not detected.                                                                                                    &                    \\
            Collimator             & An aperture for the EDS detector, blocking the X-rays with a too high angle, i.e. coming from areas outside the probed spot.                                                                                           &                    \\
            Detector window        & A barrier to keep vacuum in the EDS detector.                                                                                                                                                                          &                    \\
            Detector efficiency    & The ratio of the detected photons to the number of photons hitting the active detector are. Some X-rays pass straight through the thin SDD detectors.                                                                  & ISO15632           \\
            Solid angle ($\Omega$) & The area of the detector that is exposed to the sample. Bigger solid angle gives more counts. $\Omega = \frac{A}{r^2}$, where A is the area of the active layer and r is the distance from the sample to the detector. &                    \\
            Take-off angle         & The angle between the sample surface and the detector.                                                                                                                                                                 &                    \\
            Azimuthal angle        & The angle around the beam (seen from above) between the detector and a reference line, e.g. the chamber door.                                                                                                          &                    \\
            Working distance       & The distance between the SEM column and the sample surface.                                                                                                                                                            &                    \\
            Peak broadening        & The broadening due to the counting electronics.                                                                                                                                                                        &                    \\
            \hline
        \end{tabular}
    \end{center}
\end{table}