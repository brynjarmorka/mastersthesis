\begin{table}[pht]
    \begin{center}
        \caption{
            Definitions for various terms related to EDS detectors, \cref{theory:eds:hardware}.
            In the order they appear in the text.
        }
        \renewcommand*{\arraystretch}{1.4}
        \label{tab:eds:hardware}
        \begin{tabular}{p{2.9cm}p{12cm}}
            \hline
            \textbf{Name}                      & \textbf{Definition}                                                                                                                                                                                                                                 \\
            \hline
            SDD                                & Silicon drift detector, the most common detector.  High count rates, fast processing, and low capacitance.  See \cref{fig:eds_sdd}.                                                                                                                 \\
            Si(Li) detector                    & The older type of detectors. Much of the literature published is based on these detectors.                                                                                                                                                          \\
            Channel                            & A discrete energy interval in the measured histogram. Usually 10 eV.                                                                                                                                                                                \\
            Scale                              & Energy per channel.                                                                                                                                                                                                                                 \\
            Signal intensity                   & The number of counts stored in a channel. Signal intensity can also be expressed as counts per second, but counts is used in this work.                                                                                                             \\
            Active layer                       & The part of the detector where the detectable electron-hole pairs are generated.                                                                                                                                                                    \\
            Dead layer                         & The Si layer in the detector where generated electron-hole pair are not drifted to the anode and thus not detected.                                                                                                                                 \\
            Collimator                         & An aperture for the EDS detector, blocking the X-rays with a too high angle, i.e. coming from areas outside the probed spot.                                                                                                                        \\
            Detector window                    & A barrier to keep vacuum in the EDS detector.                                                                                                                                                                                                       \\
            Detector efficiency, $\varepsilon$ & The ratio of the detected X-rays to the number of X-rays incident on the active detector. Some X-rays pass straight through the thin SDD detectors without being detected. See \cref{fig:detector_efficiency}                                       \\
            Solid angle, $\Omega$              & The area of the detector that is exposed to the sample. Bigger solid angle gives more counts. $\Omega = \frac{A}{r^2}$, where A is the area of the active layer and r is the distance from the sample to the detector. See \cref{fig:eds_geometry}. \\
            Take-off angle                     & The angle between the sample surface and the detector. TOA increase or decrease with specimen tilt, see \cref{fig:eds_geometry}.                                                                                                                    \\
            Azimuthal angle                    & The angle around the beam (seen from above) between the detector and a reference line, e.g. the chamber door.  See \cref{fig:eds_geometry}.                                                                                                         \\
            Working distance                   & The distance between the SEM column and the sample surface. See \cref{fig:eds_geometry}.                                                                                                                                                            \\
            Peak broadening                    & The broadening due to the counting electronics.                                                                                                                                                                                                     \\
            \hline
        \end{tabular}
    \end{center}
\end{table}