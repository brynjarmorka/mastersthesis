\begin{table}[p]
    \centering
    \caption{
        The naming of characteristic X-ray lines.
        The table includes lines observed in EDS from 0.1 to 25 keV.
        The IUPAC notation is first the inner orbital, then the outer orbital.
        The Siegbahn notation is the inner shell and the lines relative intensity.
        \cref{fig:theory:xray_formation:lines} visualize some lines in the table.
        As seen for the M transitions, the Siegbahn notation does not cover all possibilities.
        Remake of Table 4.1 in Goldstein \cite{goldstein_scanning_2018}.
    }
    \label{tab:theory:naming_convention}
    \begin{tabular}{cl|cl|cl}
        \hline
        \textbf{Siegbahn} & \textbf{IUPAC} & \textbf{Siegbahn} & \textbf{IUPAC} & \textbf{Siegbahn} & \textbf{IUPAC}      \\
        \hline
        K$\alpha$$_1$     & K-L$_3$        & L$\alpha$$_1$     & L$_3$-M$_5$    & M$\alpha$$_1$     & M$_5$-N$_7$         \\
        K$\alpha$$_2$     & K-L$_2$        & L$\alpha$$_2$     & L$_3$-M$_4$    & M$\alpha$$_2$     & M$_5$-N$_6$         \\
        K$\beta$$_1$      & K-M$_3$        & L$\beta$$_1$      & L$_2$-M$_4$    & M$\beta$          & M$_4$-N$_6$         \\
        K$\beta$$_2$      & K-N$_{2,3}$    & L$\beta$$_2$      & L$_3$-N$_5$    & M$\gamma$         & M$_3$-N$_5$         \\
                          &                & L$\beta$$_3$      & L$_1$-M$_3$    & M$\zeta$          & M$_{4,5}$-N$_{2,3}$ \\
                          &                & L$\beta$$_4$      & L$_1$-M$_2$    &                   & M$_3$-N$_1$         \\
                          &                & L$\gamma$$_1$     & L$_2$-N$_4$    &                   & M$_2$-N$_1$         \\
                          &                & L$\gamma$$_2$     & L$_1$-N$_2$    &                   & M$_3$-N$_{4,5}$     \\
                          &                & L$\gamma$$_3$     & L$_1$-N$_3$    &                   & M$_3$-O$_1$         \\
                          &                & L$\gamma$$_4$     & L$_1$-O$_4$    &                   & M$_3$-O$_{4,5}$     \\
                          &                & L$\eta$           & L$_2$-M$_1$    &                   & M$_2$-N$_4$         \\
                          &                & L$l$              & L$_3$-M$_1$    &                   &                     \\
        \hline
    \end{tabular}
\end{table}