\begin{table}[phb]
    \begin{center}
        \caption{
            Definitions for various terms related to the generation of X-rays, \cref{theory:xray_formation}.
        }
        \renewcommand*{\arraystretch}{1.4}
        \label{tab:xray_generation}
        \begin{tabular}{p{3cm}p{11.6cm}}
            \hline
            \textbf{Name}                        & \textbf{Definition}                                                                                                                                                                                         \\
            \hline
            % Goldstein	&	The textbook on SEM and EDS in SEM. Refered to by almost all of the references in this work, and the co-authors have published many of the studies refered to.	\\
            Characteristic X-ray                 & An X-ray photon originating from a certain orbital transition, with a very specific energy.                                                                                                                 \\
            Incident (electron) beam             & The indident electron beam from the EM, which ionize the atoms and produce the signals detected. See  \cref{fig:characteristic_xray_formation}.                                                             \\
            Inner shell electron                 & The electron in the orbital closest to the atomic core. This is the electron that is ionized. See  \cref{fig:characteristic_xray_formation}.                                                                \\
            Outer shell electron                 & The electron from the outer orbital which is relaxed into the hole left by the ionized electron. See  \cref{fig:characteristic_xray_formation}.                                                             \\
            Atomic number, Z                     & The atomic number of the element.                                                                                                                                                                           \\
            Critical ionization energy, $E_C$    & The energy required to ionize the element of interest. Dependent on Z and the orbital to excite.                                                                                                            \\
            Secondary fluorescence               & When an X-ray photon loose energy to ionize another atom, thus creating a new characteristic X-ray.                                                                                                         \\
            Absorption edge                      & The energy where $\mu_\rho$ abruptly increase, because the energy is slightly above the $E_C$, and incoming electrons can thus ionize the respective line.  See \cref{fig:background_absorptionEdgeSi}.     \\
            Fluorescence yield, $\omega$         & The ratio of characteristic X-rays to Auger electrons, which change for the ionized shell ($\omega_K$ > $\omega_L$ > > $\omega_M$) and tend to increase with Z.  See  \cref{fig:theory:fluorescence_yield}. \\
            Ionization cross-section, $\sigma_T$ & The probability of the ionization at a certain energy for a given line. See \cref{eq:ionizationcrosssection}                                                                                                \\
            Selection rules                      & Quantum mechanical rules which specifies the allowed orbital transitions for the relaxing electron. See \cref{eq:theory:selectionrules}.                                                                    \\
            Background X-rays                    & X-rays generated by deacceleration of the electrons in the Coulomb fields of the atoms, which is noise in the analysis. See \cref{fig:background_xrays}.                                                    \\
            Relative weight                      & The relative intensity of a line compared to the strongest line in the group.                                                                                                                               \\
            Line vs. peak                        & In this work, line generally refers to the characteristic X-ray, while peak refers to the broadened signal in the spectrum.                                                                                 \\
            \hline
        \end{tabular}
    \end{center}
\end{table}